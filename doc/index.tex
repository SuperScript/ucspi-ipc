\documentclass{book}
\usepackage{sstdef}
\title{ucspi-ipc}

\begin{document}
\section{Introductoin}

\paragraph{\href{How to install ucspi-ipc}{install.html}}

\subsubsection{Servers}
\paragraph{\href{The \cmd{ipcserver} program}{ipcserver.html}}
\paragraph{\href{The \cmd{ipcrules} program}{ipcrules.html}}
\paragraph{\href{The \cmd{ipcrulescheck} program}{ipcrulescheck.html}}

\paragraph{\href{The \cmd{ipcexec} program}{ipcexec.html}}
\paragraph{\href{The \cmd{ipcexecrules} program}{ipcexecrules.html}}
\paragraph{\href{The \cmd{ipcexecrulescheck} program}{ipcexecrulescheck.html}}
\paragraph{\href{The \cmd{ipcexec-config} program}{ipcexec-config.html}}

\subsubsection{Clients}
\paragraph{\href{The \cmd{ipcclient} program}{ipcclient.html}}
\paragraph{\href{The \cmd{ipccat} program}{ipccat.html}}
\paragraph{\href{The \cmd{ipcconnect} program}{ipcconnect.html}}

\paragraph{\href{The \cmd{ipccommand} program}{ipccommand.html}}
\paragraph{\href{The \cmd{ipcdo} program}{ipcdo.html}}
\paragraph{\href{The \cmd{ipcrun} program}{ipcrun.html}}

\subsubsection{Preforking Servers}
\paragraph{\href{Compiling an \cmd{ipchandle} server}{ipchandle.html}}
\paragraph{\href{The \cmd{ipcperl} program}{ipcperl.html}}

\subsubsection{General Information}
\paragraph{\href{The ucspi-ipc environment variables}{environment.html}}
\paragraph{\href{The ucspi-ipc protocol description}{UCSPI-IPC}}

\paragraph{\href{\cmd{getpeereid}}{getpeereid.html}}
\paragraph{\href{A mailing list for ucspi-ipc and general ucspi discussion}{../lists.html\#ucspi}}


\subsection{What is ucspi-ipc?}
\cmd{ipcserver} and \cmd{ipcclient} are command-line tools for
building local-domain client-server applications.  They conform to the
UNIX Client-Server Program Interface,
\href{UCSPI}{http://cr.yp.to/proto/ucspi.txt}.

\cmd{ipcserver} listens for connections on a local-domain
stream socket, and runs a program for each connection it accepts.  The
program environment includes variables that hold the local and remote
socket addresses, and the effective user and group IDs of the process
that called \cmd{connect}.  \cmd{ipcserver} offers a concurrency limit
on acceptance of new connections, and selective handling of
connections based on client identity.

\cmd{ipcclient} requests a connection to a local-domain socket,
and runs a program.  The program environment includes a variable that
holds the local socket address.

\cmd{ipcperl} is an example of an \href{\cmd{ipchandle}}{ipchandle.html}
server.  It invokes a perl subroutine for each request.

\subsection{Features}
A service running as a privileged user under \cmd{ipcserver} can perform tasks
on behalf of nonprivileged users without setuid programs.  Clients user and
group IDs are known to the server can be logged.  Access to any service is
configurable through a standard, simple mechanism, based on the client user and
group ID.

\subsection{Operating System Support}
The ucspi-ipc package requires an implementation of \cmd{getpeereid}.
Recent Linux kernels offer sufficient basis for \cmd{getpeereid}.
Various operating systems implement a \cmd{getpeereid} system call,
including OpenBSD~3.0, FreeBSD~4.6, and AIX~5L.

\href{Patches}{../patches/intro.html} to add a \cmd{getpeereid} system
call are available for several operating systems.


\subsection{Related Software}
D.J. Bernstein created the UCSPI framework and wrote
\href{ucspi-tcp}{http://cr.yp.to/ucspi-tcp.html}.

Bruce Guenter has a protocol and package similar to ucspi-ipc called
\href{ucspi-unix}{http://untroubled.org/ucspi-unix/}
\end{document}

