\documentclass{book}
\usepackage{sstdef}
\title{ipccommand}

\begin{document}

\section{The \cmd{ipccommand} program}

\subsection{Interface}
\begin{code}%
  ipccommand \var{uid} \var{prog}
\end{code}
where \cvar{uid} is a decimal user ID and \cvar{prog} is one or more arguments.

\cmd{ipccommand} is designed to run under
\href{\cmd{ipcclient}}{ipcclient.html}.  It converts \cvar{uid} and \cvar{prog}
to a request appropriate for transmission to \href{\cmd{ipcexec}}{ipcexec.html}
and writes the result to file descriptor~7.  Thereafter, \cmd{ipccommand} reads
data from file descriptor~0 and writes it to file descriptor~7.  Simultaneously,
it reads data from file descriptor~6 and writes it to file descriptor~1.

If \cmd{ipccommand} is invoked incorrectly it complains to standard error and
exits~100.  If it encounters a read or write error, \cmd{ipccommand} complains
to standard error and exits~111.  Otherwise, it exits~0.
\end{document}

