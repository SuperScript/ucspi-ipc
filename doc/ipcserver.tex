\documentclass{book}
\usepackage{sstdef}
\title{ucspi-ipc}

\begin{document}

\section{The \cmd{ipcserver} program}

\subsection{Usage}
\begin{code}%
  ipcserver \var{opts} \var{path} \var{prog}
\end{code}
where \cvar{opts} is a series of \cmd{getopt}-style options,
\cvar{path} is a filename to use as the address of the the
local-domain socket, and \cvar{prog} is one or more arguments
specifying a program to run for each accepted connection.

\cmd{ipcserver} listens for connections from local-domain clients.
For each connection, it runs \cvar{prog}, with descriptor 0 reading
from the network and descriptor 1 writing to the network.  Before
running \cvar{prog}, \cmd{ipcserver} sets certain
\href{environment variables}{environment.html}.

\cmd{ipcserver} exits when it receives SIGTERM.

\subsection{Description}
General Options:
\begin{itemize}
\item \cmd{-q}: Quiet.  Do not print error messages.
\item \cmd{-Q}: (Default.)  Print error messages.
\item \cmd{-v}: Verbose.  Print error messages and status messages.
\end{itemize}

Connection options:
\begin{itemize}
\item \cmd{-c \var{n}}:  Do not handle more than \cvar{n} simultaneous
  connections.  If there are \cvar{n} simultaneous connections
  copies of \cmd{prog} running, defer acceptance of a new
  connection until one copy finishes.  \cvar{n} must be a positive
  integer.  The default value is 40.

\item \cmd{-x \var{cdb}}: Follow the rules compiled into \cvar{cdb} by
  \href{\cmd{ipcrules}}{ipcrules.html}.  These rules may specify setting
  environment variables or rejecting connections from bad sources.
  You can rerun \cmd{ipcrules} to change the rules while
  \cmd{ipcserver} is running.

\item \cmd{-X}: With \cmd{-x \var{cdb}}, allow connections even if \cvar{cdb}
  does not exist.  Normally \cmd{ipcserver} will drop the
  connection if \cvar{cdb} does not exist.

\item \cmd{-B \var{banner}}: Write \cvar{banner} to the network
  immediately after each connection is made.

\item \cmd{-g \var{gid}}: Switch group ID to \cvar{gid} after preparing
  to receive connections.  \cvar{gid} must be a positive integer.

\item \cmd{-u \var{uid}}: Switch user ID to \cvar{uid} after preparing
  to receive connections.  \cvar{uid} must be a positive integer.

\item \cmd{-U}: Same as \cmd{-g \$GID -u \$UID}.  Typically, \cmd{\$GID} and
  \cmd{\$UID} are set by
  \href{\cmd{envuidgid}}{http://cr.yp.to/daemontools/envuidgid.html}.

\item \cmd{-b \var{n}}: Allow a backlog of approximately \cvar{n}
  pending connections.

\item \cmd{-d}: (Default.) Delete \cvar{path} before binding it to a socket.

\item \cmd{-D}: No not delete \cvar{path} before binding it to a socket.
\end{itemize}

Data-gathering options:
\begin{itemize}
\item \cmd{-l \var{localpath}}: Do not look up the path bound to the local socket;
use \cvar{localpath} for the environment variable \cmd{\$IPCLOCALPATH}.

\item \cmd{-p}: (Default.) Look up the peer's effective user and group
IDs.  Put the results in the environment variables \cmd{\$IPCREMOTEEUID}
and \cmd{\$IPCREMOTEEGID}.

\item \cmd{-P}: Do not look up the peer's effective user and group
IDs.  Unset the environment variables \cmd{\$IPCREMOTEEUID}
and \cmd{\$IPCREMOTEEGID}.
\end{itemize}

\end{document}

