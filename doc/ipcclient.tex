\documentclass{book}
\usepackage{sstdef}
\title{ipcclient}

\begin{document}

\section{The \cmd{ipcclient} program}

\subsection{Interface}
\begin{code}%
  ipcclient \var{opts} \var{path} \var{prog}
\end{code}
where \cvar{opts} is a series of \cmd{getopt}-style options,
\cvar{path} is the filename for the client to connect to, and
\cvar{prog} is one or more arguments specifying a program to run for
each successful connection.

\cmd{ipcclient} attempts to connect to the local-domain socket
associated with the filename \cvar{path}.  For each connection, it runs
\cvar{prog}, with descriptor 6 reading from the network and descriptor
7 writing to the network.  Before running \cvar{prog}, \cmd{ipcclient}
sets certain \href{environment variables}{environment.html}.

\subsection{Options}
General Options:
\begin{itemize}
\item \cmd{-q}: Quiet.  Do not print error messages.
\item \cmd{-Q}: (Default.)  Print error messages.
\item \cmd{-v}: Verbose.  Do not print error messages and status messages.
\end{itemize}

Connection options:
\begin{itemize}
\item \cmd{-p \var{localpath}}: Bind the local socket to
  \cvar{localpath} before attempting a connection.  Without this
  option, do not bind any filename to the local socket.
\end{itemize}

Data-gathering options:
\begin{itemize}
\item \cmd{-l \var{localname}}: Do not look up the name associated
  with the local socket; use \cvar{localname} for the environment
  variable \cmd{\$IPCLOCALPATH}.
\end{itemize}

\end{document}
