\documentclass{book}
\usepackage{sstdef}
\title{ucspi-ipcperl}

\begin{document}

\section{The \cmd{ipcperl} program}

\subsection{Usage}
\begin{code}%
  ipcperl \var{opts} \var{file} \var{prog}
\end{code}
where \cvar{opts} is a series of \href{ipchandle}{ipchandle.html} server
options, \cvar{file} names a file containing perl code, and \cvar{prog}
is one or more arguments specifying a perl subroutine plus arguments to
run for each accepted connection.

\cmd{ipcperl} forks the requested number of children, each listening for
IPC client connections.  Before handling any requests, each child parses
and runs the perl code in \cvar{file}.  The file should end with a true
expression, like a module.

To handle a request, a child process executes \cvar{prog} as a perl
subroutine, with file descriptor~0 reading from the network and file
descriptor~1 writing to the network.  Before handling the request, the
child sets certain environment variables, a la
\href{\cmd{ipcserver}}{ipcserver.html}.

Each invocation of \cvar{prog} handles a single request.  It is called
within a loop, with one iteration per request, and therefore must
release any resources allocated to handle a particular request.

If \cvar{prog} exits while handling a request, \cmd{ipcperl} will
start a new child process.


\subsection{Configuration}
Edit the following files in \cmd{src/} as necessary for your
installation.  Unless you use modules that require \cmd{xs} support or a
nonstandard version of perl, the defaults should suffice.
\begin{itemize}
\item\cmd{ipcperl.c}
  If your server uses a module like \cmd{DBI.pm} you may need to add the
  requisite \cmd{xs} code.

\item\cmd{conf-ldperl}
  Determine options required to load \cmd{ipcperl}.  By default,
  these are calculated automatically.  Using \cmd{xs} code in
  \cmd{src/ipcperl.c} may require manual option setting in
  \cmd{src/conf-ldperl}.

\item\cmd{conf-ccperl}
  Determine options required to compile \cmd{ipcperl}.  By default,
  these are calculated automatically.

\item\cmd{conf-perl}
  How to invoke perl to calculate options automatically in
  \cmd{conf-ccperl} and \cmd{conf-ldperl}.
\end{itemize}


\subsection{Known Problems}

CDB\string_File: iterative lookups succeed but random lookups fail.
Each \cmd{ipchandle} server uses the original
\href{cdb}{http://cr.yp.to/cdb.html} library for access control.  The 
\href{CDB_File-0.92}{http://search.cpan.org/author/MSERGEANT/CDB_File-0.92}
package redefines \cmd{cdb\string_findnext} without declaring it static.  The
linker may select the wrong definition.  Solution: add \cmd{static} to
the declaration of \cmd{cdb\string_findnext} in CDB\string_File.xs.

\end{document}
