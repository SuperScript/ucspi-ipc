\documentclass{book}
\usepackage{sstdef}
\title{ipcexec}

\begin{document}

\section{The \cmd{ipcexec} program}

\subsection{Interface}
\begin{code}%
  ipcexec \var{cdb}
\end{code}
where \cvar{cdb} is a set of rules compiled by
\href{\cmd{ipcexecrules}}{ipcexecrules.html}.

\cmd{ipcexec} reads a user ID \cvar{USERID} and a program \cvar{prog} from file
descriptor 0, and reads the remote user ID and group ID from the environment
variables \cmd{IPCREMOTEEUID} and \cmd{IPCREMOTEEGID}.  If the rules in
\cvar{cdb} permit the remote user to execute \cvar{prog} as \cvar{USERID}, then
\cmd{ipcexec} changes its effective user and group ID to match \cvar{USERID},
and executes \cvar{prog}.

By default, \cmd{ipcexec} redirects standard error to standard output before
executing \cvar{prog}.  If the \cmd{\$IPCERROUT} environment variable is set to
\cmd{0}, then \cmd{ipcexec} does not redirect standard error before executing
\cvar{prog}.

If the rules in \cvar{cdb} deny execution of the request, then \cmd{ipcexec}
silently exits~100.


After processing the matching rule from \cvar{cdb} and before executing the
request, \cmd{ipcexec} checks the following environment variables:
\begin{itemize}
\item \cmd{\$IPCERROUT}:
If decimal nonzero, \cmd{ipcexec} redirects standard error to standard output
before executing \cvar{prog}.

\item \cmd{\$IPCUID}:
If set, then \cmd{ipcexec} sets its effective user ID to \cmd{\$IPCUID} before
executing \cvar{prog}.  This environment variable setting overrides \cvar{USERID}.

\item \cmd{\$IPCGID}:
If set, then \cmd{ipcexec} sets its effective group ID to \cmd{\$IPCGID} before
executing \cvar{prog}.  This environment variable setting overrides \cvar{USERID}.
\end{itemize}
\end{document}

