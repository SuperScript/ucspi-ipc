\documentclass{book}
\usepackage{sstdef}
\title{ucspi-ipc}

\begin{document}

\section{How to install \package}

\package{} comes with NO WARRANTY.

\subsection{System requirements}
\package{} works only under UNIX.  In addition, \package{} requires
an implementation of \href{\cmd{getpeereid}}{getpeereid.html}.

This package implements \cmd{getpeereid} for recent Linux kernels.
Some systems offer \cmd{gepeereid} as system call, including
\href{OpenBSD~3.0}{http://www.openbsd.org},
\href{FreeBSD~4.6}{http://www.freebsd.org} and
\href{AIX~5L}{http://www-1.ibm.com/servers/aix/index.html}.
Several \href{patches}{../patches/index.html} are also available to add
a \cmd{getpeereid} system call to older BSD operating system versions.

On systems that lack \cmd{getpeereid}, \package{} runs, but
\href{\cmd{ipcserver}}{ipcserver.html} generates an error if \cmd{getpeereid}
is called, i.e. when invoked without the \cmd{-P} switch.  The package
regression tests fail on platforms without \cmd{getpeereid}.

\subsection{Installation}
If necessary, create a \cmd{/package} directory:
\begin{code}%
  mkdir /package
  chmod 1755 /package
  cd /package
\end{code}

Download the \package{} package.  The most recent \package{} package is
\href{\package~\version}{\package-\version.tar.gz}.  Unpack \package{} in
\cmd{/package}:
\begin{code}%
  gunzip \package-\version.tar
  tar -xpf \package-\version.tar
  rm \package-\version.tar
  cd host/superscript.com/net/\package-\version
\end{code}

Compile the package:
\begin{code}%
  package/compile
\end{code}

Run some tests:
\begin{code}%
  package/rts
\end{code}
The output should be empty.  Tests will fail on systems that lack
\cmd{getpeereid} support.

Install the package:
\begin{code}%
  package/install
\end{code}

Commands are installed in the
\cmd{/package/host/superscript.com/command} directory.
For compatibility with prior versions, commands are also installed in
the \cmd{/usr/local/bin} directory.

Report successful installation:
\begin{code}%
  package/report
\end{code}


\subsection{Package subsets}
To restrict the compile or install to a subset of the entire package,
supply the name of a subset on the command line:
\begin{code}%
  package/compile \var{subset}
  package/rts \var{subset}
  package/install \var{subset}
\end{code}

To exclude a subset from an operation, add a hyphen as the first
command-line argument:
\begin{code}%
  package/compile - \var{subset}
  package/rts - \var{subset}
  package/install - \var{subset}
\end{code}

Valid subsets in this package are \cmd{base} and \cmd{ipcperl}.
\end{document}

