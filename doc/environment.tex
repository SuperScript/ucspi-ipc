\documentclass{book}
\usepackage{sstdef}
\title{ucspi-ipc}

\begin{document}

\section{The ucspi-ipc environment variables}

\cmd{\$PROTO} is the string \cmd{IPC}.  Both \cmd{ipcserver} and
\cmd{ipcclient} set this environment variable.

\cmd{\$IPCLOCALPATH} is the file name associated with the local
socket.  Both \cmd{ipcserver} and \cmd{ipcclient} set this environment
variable.

\cmd{\$IPCREMOTEPATH} is the path associated with the remote socket.
Only \cmd{ipcserver} sets this environment variable.  If the remote
socket is not bound to a path, then this environment variable is set,
but empty.  Beware that \cmd{\$IPCREMOTEPATH} is under the control of
the remote user, and can contain arbitrary characters.

\cmd{\$IPCREMOTEEUID} is the effective user id of the client process
that called \cmd{connect}, in decimal.  Only \cmd{ipcserver} sets this
environment variable.

\cmd{\$IPCREMOTEEGID} is the effective group id of the client process
that called \cmd{connect}, in decimal.  Only \cmd{ipcserver} sets this
environment variable.
\end{document}
