\documentclass{book}
\usepackage{sstdef}
\title{ucspi-ipc}

\begin{document}

\section{The \cmd{ipcrun} program}

\subsection{Interface}
\begin{code}%
  ipcrun \var{prog}
\end{code}
where \cvar{prog} is one or more arguments specifying a program to run via
\href{\cmd{ipccommand}}{ipccommand.html}.

\cmd{ipcdo} uses \cmd{ipccommand} and \href{\cmd{ipcclient}}{ipcclient.html} to
connect to the \href{\cmd{ipcexec}}{ipcexec.html} server (specified at compile
time).  It makes request a request to execute \cvar{prog} with the highest
privilege permitted by the server.

\cmd{ipcrun} sends any input it receives to the \cmd{ipcexec} server, and prints
any data it receives.  The usage above is equivalent to the following
\href{\cmd{ipcdo}}{ipcdo.html} invocation
\begin{code}%
  ipcdo root \var{prog}
\end{code}
\end{document}

